\documentclass[a5paper]{book}
\usepackage{amsmath,amsthm,amssymb}
\usepackage[integrals]{wasysym} % integrals
\usepackage{mathtext}
\usepackage[T1,T2A]{fontenc}
\usepackage[utf8]{inputenc}
\usepackage[english,russian]{babel}
\usepackage[margin=1in]{geometry}
\usepackage{geometry}
\usepackage{indentfirst}
\usepackage{soul}
\usepackage{microtype}
\usepackage{fancyhdr}

\usepackage{ragged2e} % выравнивание по ширине
\justifying

\pagenumbering{gobble} % удаляет номер страницы внизу страницы

\newcommand{\eqdef}{{=\joinrel=}} % длинное "равно"

\begin{document}

% отступы от формул
\setlength{\abovedisplayskip}{3pt}
\setlength{\abovedisplayshortskip}{3pt}
\setlength{\belowdisplayskip}{3pt}
\setlength{\belowdisplayshortskip}{3pt}

\lhead{\textsl{\scriptsize 346}}
\chead{\textsl{\scriptsize {\it Глава} 25}}
\pagestyle{fancy}

\fontdimen2\font=0.4em % расстояние между словами

\noindent Следовательно, \begin{align*} \int\limits_a^b{f(x)\,dx}\;существует\;и = L. \end{align*}
\indent \textbf{Теорема 369.} \textit{Каждая функция, непрерывная в} [\textit{a}, \textit{b}], \textit{интегрируема от a до b.}
\\ \indent Д\kern+0.2em о\kern+0.2em к\kern+0.2em а\kern+0.2em з\kern+0.2em а\kern+0.2em т\kern+0.2em е\kern+0.2em л\kern+0.2em ь\kern+0.2em с\kern+0.2em т\kern+0.2em в\kern+0.2em о.
Пусть $f(x)$ \textemdash \, функция, удовлетворяющая условию теоремы, и пусть задано $\delta > 0$. В силу теоремы 154, существует $\varepsilon > 0$ такое, что
\begin{gather*}
    |f(x) \, \textemdash \, f(\beta)| < \frac{\delta}{2(b \, \textemdash \, a)} \, при \; a \leqslant \alpha \leqslant b, \: a \leqslant \beta \leqslant b, 
    \\ |\alpha \, \textemdash \, \beta| \leqslant \varepsilon.
\end{gather*}
Если, следовательно, каждое $e_v < \varepsilon $, то каждое $$ s_v \leqslant \frac{\delta}{2(b-a)} $$
и
$$ \sum es \leqslant \frac{\delta}{2(b-a)} \sum e \; \eqdef \; \frac{\delta}{2} < \delta.$$
Тем самым выполнение условия Римана проверено (что, согласно теореме 368, в полном объёме даже не было необходимо). \vspace{2mm} \\
\indent \textbf{Определение 85.}\textit{ Функция $f(x)$, определённая в \textnormal{[}\textit{a},\, \textit{b}\textnormal{]}, называется там монотонной, если для $ a \leqslant \alpha < \beta \leqslant b$ всегда}
$$ f(\alpha) \leqslant f(\beta), $$
\textit{либо всегда}
$$ f(\alpha) \geqslant f(\beta). $$
\textit{В первом случае $f(x)$ называют монотонно не убывающей, во втором \textemdash \, монотонно не возрастающей.} \\
\indent (Определения 71 и 72 исключают знак равенства между $f(\alpha)$ и $f(\beta)$). \vspace{2mm} \\
\indent \textbf{Теорема 370.}\textit{ Каждая монотонная в \textnormal{[}$a, \; b$\textnormal{]} функция интегрируема от $a$ до $b$.}

\newpage

\lhead{}
\chead{\textsl{\scriptsize {\it Понятие определённого интеграла}}}
\rhead{\textsl{\scriptsize 347}}
\pagestyle{fancy}

\indent Д\kern+0.2em о\kern+0.2em к\kern+0.2em а\kern+0.2em з\kern+0.2em а\kern+0.2em т\kern+0.2em е\kern+0.2em л\kern+0.2em ь\kern+0.2em с\kern+0.2em т\kern+0.2em в\kern+0.2em о.
Пусть $f(x)$ \textemdash \, функция, удовлетворяющая условию теоремы. В первом случае определения 85 имеем
$$ l_v \eqdef f(a_v), \kern+0.7em \lambda \eqdef f(a_{v-1}), \kern+0.7em s_v \eqdef f(a_v) \, \textemdash \, f(a_{v-1}),$$
и для $e_v < \varepsilon$ получаем
$$ \sum es \leqslant \varepsilon \sum s \; \eqdef \; \varepsilon(f(b) \, \textemdash \, f(a)). $$
\indent Во втором случае
$$ l_v \eqdef f(a_{v-1}), \kern+0.7em \lambda \eqdef f(a_v), \kern+0.7em s_v \eqdef f(a_{v-1}) \, \textemdash \, f(a_v),$$
и для $e_v < \varepsilon$ получаем
$$ \sum es \leqslant \varepsilon \sum s \; \eqdef \; \varepsilon(f(a) \, \textemdash \, f(b)). $$
Таким образом, каково бы ни было $\delta > 0$, если $e_v < \varepsilon$, где
$$ \varepsilon \, \eqdef \, \frac{\delta}{|f(b) \, \textemdash \, f(a)| + 1}, $$
то имеем
$$ \sum es < \delta.$$ 
\indent \textbf{Теорема 371.}\textit{ Каждая ограниченная в \textnormal{[}a,\, b\textnormal{]} функция, для которой число мест разрыва внутри этого интервала не бексонечно, интегрируема от a до b.}
\\ \indent Д\kern+0.2em о\kern+0.2em к\kern+0.2em а\kern+0.2em з\kern+0.2em а\kern+0.2em т\kern+0.2em е\kern+0.2em л\kern+0.2em ь\kern+0.2em с\kern+0.2em т\kern+0.2em в\kern+0.2em о.
Пусть $f(x)$ \textemdash \, функция, удовлетворяющая условию теоремы, и
$$ |f(x)| < c \; \; в \; \; [a,\;\; b].$$
Обозначим $a, b$ и возможные внутренние места разрыва функции $f(x)$ через
$$\eta_k, \;\; 0 \leqslant k \leqslant m,\; k \;\; целое, $$
так, чтобы 
\begin{gather*}
    \eta_{k-1} < \eta_k \;\; при \;\; 1 \leqslant k \leqslant m,
    \\ \eta_0 \, \eqdef \, a, \; \; \eta_m \, \eqdef \, b
\end{gather*}
\indent Пусть $\delta > 0$. Положим 
\begin{gather*}
    \underset{1 \, \leqslant \, k \, \leqslant \, m}{\textnormal{Min}}(\eta_k \textemdash \, \eta_{k-1}) \eqdef \zeta,
    \\ \gamma \; \eqdef \; \textnormal{Min}\left( \frac{\delta}{8mc}, \; \frac{\zeta}{3}\right).
\end{gather*}

\end{document}